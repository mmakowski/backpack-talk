\documentclass{beamer}

\usepackage[UKenglish]{isodate}
\usepackage[utf8]{inputenc}
\usepackage[T1]{fontenc}
\usepackage{textcomp}
\usepackage{listings}
\usepackage{lmodern}

\usecolortheme{dove}
\setbeamertemplate{navigation symbols}{}

\begin{document}

\title{Modularity and Backpack}
\author{Maciek Makowski (@mmakowski)}

% -------------------------------------------
\frame{\titlepage}

% -------------------------------------------
\begin{frame}{MirageOS}
TODO picture of unikernel
\end{frame}

% -------------------------------------------
\begin{frame}{MirageOS}
TODO picture of HTTP app with dependencies
\end{frame}

% -------------------------------------------
\begin{frame}{Exhibit 1}{MirageOS}
\begin{quote}
The compiler heavily emphasises static type checking, and the resulting binaries are fast native code with no runtime type information and the module system is among the most powerful in a general-purpose programming language in terms of permitting flexible and safe code reuse and refactoring.
\end{quote}
-- Technical Background of MirageOS
\end{frame}

% -------------------------------------------
\begin{frame}[fragile]{ML Modules}{Structures}
\begin{lstlisting}[language=ML]
structure IntInteger =
  struct
    type integer = int
    val zero = 0
    val succ n = n + 1
    fun add a b = a + b
    fun mul a b = a * b
  end
\end{lstlisting}
\end{frame}

% -------------------------------------------
\begin{frame}[fragile]{ML Modules}{Signatures}
\begin{lstlisting}[language=ML]
signature INTEGER =
  sig
    type integer
    val zero: integer
    val succ: integer -> integer
    fun add: integer -> integer -> integer
    fun mul: integer -> integer -> integer
  end
\end{lstlisting}
\end{frame}

% -------------------------------------------
\begin{frame}[fragile]{ML Modules}{Functors}
\begin{lstlisting}[language=ML]
functor RationalFun(I: INTEGER) =
  struct
    type rational = I.integer * I.integer
    fun nom (n, _) = n
    fun denom (_, d) = d
    fun add (n1, d1) (n2, d2) =
      (I.add (I.mul n1 d2) (I.mul n2 d1),
       I.mul d1 d2)
  end

structure IntRational = RationalFun(IntInteger)
\end{lstlisting}
\end{frame}

% -------------------------------------------
\begin{frame}{ML Modules}
TODO: Mirage app diagram
\end{frame}

% -------------------------------------------
\begin{frame}[fragile]{Exhibit 2}{tagstream-conduit}
\begin{lstlisting}[language=Haskell]
data Dec builder string = Dec
  { decToS     :: builder -> string
  , decBreak   :: (Char -> Bool) -> string ->
                  (string, string)
  , decBuilder :: string -> builder
  , decDrop    :: Int -> string -> string
  , decEntity  :: string -> Maybe string
  , decUncons  :: string -> Maybe (Char, string)
  }
\end{lstlisting}
\end{frame}

% -------------------------------------------
\begin{frame}[fragile]{Exhibit 3}{tagsoup}
\begin{lstlisting}[language=Haskell]
class (Typeable a, Eq a) => StringLike a where
    empty      :: a
    cons       :: Char -> a -> a
    uncons     :: a -> Maybe (Char, a)
    toString   :: a -> String
    fromString :: String -> a
    -- [...]
\end{lstlisting}
\end{frame}

% -------------------------------------------
\begin{frame}[fragile]{Backpack}{Signatures}
\begin{lstlisting}[language=Haskell]
module IntegerSig where

data Integer

zero :: Integer
succ :: Integer
add  :: Integer -> Integer -> Integer
mul  :: Integer -> Integer -> Integer
\end{lstlisting}
\end{frame}

% -------------------------------------------
\begin{frame}[fragile]{Backpack}{Modules}
\begin{lstlisting}[language=Haskell]
module Rational where

import qualified IntegerSig as I

data Rational = R I.Integer I.Integer

nom (R n _) = n
denom (R _ d) = d
add (R n1 d1) (R n2 d2) =
  R (I.add (I.mul n1 d2) (I.mul n2 d1))
    (I.mul d1 d2)
\end{lstlisting}
\end{frame}

\end{document}
